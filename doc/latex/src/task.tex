\documentclass[14pt, a4paper]{extreport}

\usepackage[T2A, T1]{fontenc}
\usepackage[utf8]{inputenc}
\usepackage[main=russian, english]{babel} 
\usepackage{graphicx}
\usepackage[margin=2cm]{geometry}
\usepackage{tabularx}
\usepackage{soul}
\usepackage{hyperref}
\usepackage{microtype}
\usepackage{enumitem}
\pagestyle{empty}

\hypersetup{
    colorlinks=false,
}

\begin{document}

\begin{center}
    \vspace{0.5em}
    {{Форма задания по курсовой работе (курсовому проекту)}\\ 
    {Белорусский государственный университет}}
\end{center}

\noindent
\begin{tabularx}{\textwidth}{@{}lX@{}}
   {Факультет} & \ul{радиофизики и компьютерных технологий} \\ 
   {Кафедра} &  \ul{физики и аэрокосмических технологий} \\ 
\end{tabularx}

\vspace{1em}
\begin{center}
    \textbf{ЗАДАНИЕ ПО КУРСОВОЙ РАБОТЕ (КУРСОВОМУ ПРОЕКТУ)}
\end{center}

\noindent 
\begin{enumerate}
\item {Студент} \ul{Глеба Евгений Михайлович}
\item {Тема} \ul{МЕТОДЫ ОЦЕНКИ ВЛИЯНИЯ КОСМИЧЕСКОЙ ПОГОДЫ НА БОРТОВЫЕ СИСТЕМЫ АКТИВНЫХ СПУТНИКОВ} 
\item {Срок представления курсовой работы (курсового проекта) к защите} \ul{28 мая 2024 г.} 
\item {Исходные данные к курсовой работе (курсовому проекту)} (при необходимости)
    \begin{enumerate}[label=\arabic{enumi}.\arabic*]
        \item \ul{Redouane Boumghar и др. «Behaviour-based anomaly detection in spacecraft
        using deep learning». В: Libre Space Foundation (2024)} 
        \item \ul{J. C. Green, J. Likar и Y. Shprits. «Impact of space weather on the satellite
        industry». В: Advancing Earth and space science 15 (2017)} 
        \item \ul{R. Killick, P. Fearnhead и I. A. Eckley. «Optimal detection of changepoints
        with a linear computational cost». В: J. Amer. Statist. Assoc. 107 (2012)} 
        \item \ul{R. Boumghar и др. «Enhanced awareness in space operations using multipurpose dynamic network analysis». В: Space Operations: Inspiring Humankind’s Future. Springer International Publishing, 2018} 
    \end{enumerate}
\item {Содержание (структура) курсовой работы (курсового проекта):}
    \begin{enumerate}[label=\arabic{enumi}.\arabic*]
        \item \ul{Постановка цели и задач исследования (введение)} 
        \item \ul{SatNOGS: Открытая инфраструктура для мониторинга низкоорбитальных спутников} 
        \item \ul{Платформа Polaris ML} 
        \item \ul{Анализ перезагрузок основного процессора и граф связности спутника GRIFEX} 
    \end{enumerate}
\end{enumerate}

\vspace{2em}

\noindent
\begin{tabularx}{\textwidth}{@{}|X|X@{}|}
    \hline
    \multicolumn{1}{|c|}{\textbf{Содержание задания}} & \multicolumn{1}{c|}{\textbf{Сроки выполнения}} \\
    \hline
    Комплексная оценка влияния космической погоды на работу бортовой электроники космических аппаратов & 5.02.2024-15.03.2024 \\
    \hline
    Разработка и оптимизация платформы Polaris ML для анализа больших данных телеметрии & 16.03.2024-15.04.2024 \\
    \hline
    Проверка эффективности платформы Polaris ML на реальных данных космического аппарата GRIFEX & 16.04.2024 - 25.05.2024 \\
    \hline
\end{tabularx}

\vspace{2em}

% \begin{tabularx}{XX}
% Содержание задания (наименование структурных элементов, этапов работы) \centering & Сроки выполнения \centering \\
%          &          \\
%          &          \\
%          &          \\
% \end{tabularx}

\noindent
\begin{tabularx}{\textwidth}{@{}l@{\hspace{0.5cm}}l@{\hspace{1cm}}X@{}}
  {Руководитель курсовой}  \\
  {работы (курсового} &  & \\ 
  {проекта)} & \hrulefill & \hrulefill \\
                                & \centering (подпись, дата) & \centering (инициалы, фамилия)
\end{tabularx}

\vspace{3em}

\noindent
\begin{tabularx}{\textwidth}{@{}l@{\hspace{5pt}}X@{}}
    Задание принял к исполнению & \hrulefill \\
    & \centering (подпись, дата)
\end{tabularx}

\vspace{1em}

\noindent {Проинформирован о недопустимости привлечения третьих лиц к выполнению курсовой работы (курсового проекта), плагиата, фальсификации или подлога материалов.} 

\vspace{2em}

\begin{tabularx}{\textwidth}{XX}
  \hrulefill & \hrulefill \\
  \centering (подпись) & \centering (инициалы, фамилия обучающегося) 
\end{tabularx}

\vspace{1em}

\noindent \textbf{ПРИМЕЧАНИЕ.} Допускается дополнять или исключать пункты в бланке задания. 

\end{document}